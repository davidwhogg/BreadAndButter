% This file is part of the BreadAndButter project.
% Copyright 2013 David W. Hogg (NYU) and any other authors.

\documentclass[letterpaper,12pt,preprint]{aastex}

\newcommand{\project}[1]{\textsl{#1}}
\newcommand{\sdss}{\project{SDSS}}
\newcommand{\given}{\,|\,}
\newcommand{\transpose}[1]{{#1}^{\mathsf{T}}}
\newcommand{\inverse}[1]{{#1}^{-1}}

\begin{document}

\title{Inferring the gravitational potential of the Milky Way with measurements of \emph{just two stars}}
\author{DWH, APW, KVJ, others}

\begin{abstract}
Informative phase-space structures, and in particular those created by
disrupting stellar systems or revealed by chemical tags, may deliver
very precise measures of the gravitational potential in the Milky Way
Halo.  Here we show that even a single pair of stars---two stars that
are known (for some non-kinematic reason) to be likely to be
associated with one another at birth---could provide a significant
constraint on the potential.  The inference is based on a
probabilistic generative model that evolves forward in time to the
present day from a disruption event in the past.  The time,
six-volume, and phase-space location of the origin event are nuisance
parameters in the model and marginalized away.  The method makes no
assumption of integrability, works with finite or even large
observational uncertainties, does not require all dimensions of phase
space to be observed, handles non-zero probability that the two stars
are not in fact associated, and generalizes naturally to larger
numbers of stars and multiple independent structures.  Applications to
the GD-1 cold stellar stream and current surveys for RR Lyrae stars in
the Halo are discussed.
\end{abstract}

\keywords{
  this ---
  that ---
  Milky Way
}

\section{Introduction}

...Cold streams contain tons of information.

...GD-1 was fit as if it highlighted an orbit.  That is known to be
wrong.

...In the halo, any phase-space structure might be long-lived.
Chemical tagging could in principle illuminate it.

...Chemical tagging will always only give probabilistic information.
But what if it delivers somewhat confident information about small
stellar families.

...Actions, angles, integrable, not, etc.  Can integrate forward in
any potential, even a time-varying one.

\section{The model}

\section{Experiments}

\section{Discussion}

\acknowledgements
Binney, Rix, Sanders

\end{document}
